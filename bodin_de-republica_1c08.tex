
\def\filedate{2014/10/10}
\def\fileversion{0.1}%
%
%
\newif\ifproof % <-- generic conditional for proof-reading hooks
\prooftrue
%
%
\documentclass[12pt,twoside]{memoir}

% low-level basics ===================================================
\usepackage{ifxetex,etoolbox}
\usepackage[svgnames]{xcolor}% load before bidi/polyglossia
\usepackage{xparse}


% Git :: gitinfo2 ====================================================

\usepackage[mark]{gitinfo2}%
\renewcommand*{\gitMark}{Branch: \gitBranch\,@\,\gitAbbrevHash{}
  (\gitAuthorDate)}%
\renewcommand*{\gitMarkFormat}{\color{gray}\footnotesize\ttfamily}


% My packages ========================================================
\usepackage{fonts+lang}
\usepackage{commands-generic}
\usepackage{page-setup}
\usepackage{terminology}
\usepackage{texts-law}


\begin{document}% ====================================================


\setcounter{chapter}{7}
\chapter{On the Right of Majesty}


\Maiestas\ is the \sumpot\ over citizens and subjects, and it is freed
from the laws,%
\footnote{Cf. \dig{1.3.31}.} %
which the Greeks call
  % switch to Greek input: C-x RET C-\ greek‐ibycus4
\grk{ἄκραν ἐξουσίαν},%
\footnote{Unlimited power?} %
\additio{or} sometimes %
\grk{κυρίαν ἀρχλὼ}% CHECK
\footnote{TODO} %
and \grk{κὺριον ωολέτευμα},% CHECK
\footnote{TODO} %
\mnote*{Foundation of the Commonwealth} %
the Italians \textitalian{\emph{Segnoria}},%
\footnote{\emph{Signoira} may be translated as lordship or
  dominion.} %
%
% CHECK THE HEBREW
\additio{and}
the Hebrews \texthebrew{.תרמר שכט}, %
%
that is, \maiusimperium.  For \maiestas, says Festus, is named from
\qq{magnitude}[.]  \Maiestas\ should have been defined from the start,
which none of the philosophers or jurists did define, since it seems
that there is yet \page{79a} nothing greater or more necessary for
understanding the nature of the \Respublica.  And since we defined
\Respublica\ above
% where?
as a proper \platin{rectam} \gubernatio\ of several families and
things common among them with the highest and perpetual \potestas, it
should be explained why it comes from the appellation of \qq{highest
  and perpetual \potestas}[.]  We said it must be perpetual because
that a  \sumpot\ over \platin{in} citizens can come to be for
one \additio{person} or many; and yet it may not be allotted as
perpetual, but \additio{only} for a brief time: at the end of which,
they relinquish \platin{abdicant} the  \sumpot.  Therefore,
they cannot be called the highest \principes, but rather \custodes\ of
the  \sumpot\ and \imperium\ for so as long as the highest
\princeps\ or \populus\ reclaims the entrusted \platin{depositum}
\imperium\ of which they themselves are the truest possessors and
lords---\page{79b}%
not differently than those who gave their things for a loan or
as a pledge:%
%
\mnote{l.\ qui pignori, de usucap.; l.\ quod meo, de acquir.\ posess.%
 % \addtocounter{footnote}{-1}%
  \footnotemark}%
[\dig{41.3.33.4}(!); \hldig{41.2.18}] %
or, \additio{for as long as} they permitted their \iurisdictio\ or
\imperium\ to be enjoyed by another (whether for a certain time or for
a \lemph{precarium}%
\footnote{\dig{43.26.1 pr.}: \lemph{Precarium} is what is granted to
  one for him who seeks it with prayers to use for as long as he who
  granted it allows it.}%
% pr. Precarium est, quod precibus petenti utendum conceditur tamdiu,
% quamdiu is qui concessit patitur.
), they do not cease to be the arbiters and possessors of their
\potestas\ and \iurisdictio.%
\mnote{l.\ more.\ de iurisdic., l.\ et quia eod.\footnotemark}%
[\dig{2.1.5}; \hldig{2.1.6}.] %
Thus the Jurisconsult%
\footnote{This epithet usually refers to Ulpian in the same way
  Aristotle was known as \qq{the Philosopher}[.]} %
said that a prefect of Augustus is to return the entrusted \imperium\
once his magistracy has come to an end.% exacto magistratu
\mnote{l.\ 1, de offi.\ praefecti Augustalis\footnotemark}%
[\dig{1.17.1}.] %
Nor does it matter if a greater or lesser \potestas\ be assigned:
otherwise, if the highest \sumpot, granted as a \FIX{precarious
  bequest} of the \princeps\ were called \maiestas, he would be able
to use that power by his own \ius\ \platin{suo iure} against a
\princeps\ to whom nothing would remain beyond the empty name
\princeps: a servant, moreover, would command his lord.  Nothing more
absurd than that can be thought, for the person of the \princeps\ is
always excepted in every concession of \imperium\ to magistrates or
private individuals:%
\mnote{l.\ ult., qui satisdate; \CHECK{cor.\ ser., de pot. regia.}
  9. 17.\footnotemark}%
[\dig{2.8.16}.] %
however much \page{79c} \imperium\ there be that is assigned to
another, it is still less than what he has reserved to himself by his
\ius\ \maiestas.  Nor is it ever supposed that he has been despoiled
of his \sumpot; rather, it is supposed that he can take cognizance%
\mnote{l.\ Iudicium soluitur, de Iudic.\ l. solet de Iurisdic.} %
of those things which he assigned to the magistrates or
superintendents, either by anticipation, \CHECK{joint exercise}, or
\CHECK{summons}, and to take \additio{back} by force%
\mnote{Alexand.\ in l.\ ult. de iurisdict.; Pano. in cap.\ Pastoralis,
  de offi. ord.; Innocentius \& Felin.\ in cap. \CHECK{cum}
  ecclesiarum, eod.\footnotemark}[TODO] %
all \potestas\ \concessum\ to the magistrates.


From these \additio{reasons}, it happens that none have the
\iuramaiestatis: not the \Dictator{} of the Romans, not the Harmost of
Spartans, not the Esymnete of the Salonikans, not the Archus of
Maltese, not the Balia of the Florentines (when they enjoyed
\platin*{uterentur} popular \potestas), nor those who are called
regents of a \regnum\ by men today, % nostris hominibus
nor any other \magistratuspl\ or \curatores\ who have \sumpot\ (though
it be not allotted perpetually by concession of the \princeps\ or
\populus). \page{79d} Nor, even, did those ancient dicators have the
\sumius---even if they were appointed \platin{dicebantur} by the
\emph{noblest \lex}%
\footnote{TODO}%
---from whom one could not appeal,% provocari
\mnote{Festus in verbo optima lege. Plutar.\ in
  quaestion. Rom.\footnotemark}%
[TODO] % see also berger pdf331
and on whose creation all \magistratuspl\ resign \additio{their
  offices} until, \CHECK{by the sacred measure proposed [one] June},
the Tribunes of the \plebs\ were made sacred \custodes\ of popular
liberty, who, on the creation of the \Dictator{}, had the free \ius\
of intercession so that, if it were appealled from the \Dictator{},
the Tribunes could convene % cogerent
the \plebs, among whom there was contention % certabatur
about the appeal. % provocatio
For the \Dictator{} Papirius condemned to death Fabium Maximus I, and
Fabius Maximus II also condemned to death Minutius, \additio{both}
\magistri\ of the light cavalry \platin{minutium equitum}, because
they had fought against the command of the \Dictatores.  Even so, they
were freed on appeal \platin{ex provocatione} by judgment of the
\populus.  For, as Livy says:%
\mnote{9 lib.\ 7\footnotemark}[TODO] %
\qq{First, Fabius' father said, I call upon the Tribues, and I appeal
  to the \populus, who are more able than your \dictatura, to whom the
  \rex\ Tullus Hostilius yielded.}  It should be clear from these
words that he was neither \princeps\ nor chief \magistratus\ (as many
have thought), but a \curator\, whom they call our \page{80a}
\commissarius; nor that anything else was assigned to him beyond
taking charge of war-making \platin{curationem belli gerendi}, curbing
sedition, \CHECK{fixing the Republic} \platin{Reipublicae
  constituendae}, creating \magistratuspl, or \CHECK{helming the
  tiller} \platin{clavi figendi}.  \Maiestas, on the other hand, is
not delimited by a greater \potestas, any other \leges, or time.  But
not even the \emph{decemvirs} had a \iusmaiestatis\ for the
promulgation of \leges\ \platin{legum ferendarum}, even though they
had the \sumpot, freed also from the \leges, and even though all
\magistratuspl\ resign \additio{their office} at the creation of the
\emph{decemvirs}, since their \imperium\ takes its limit % finem
from the \leges\ of the Twelve Tables, which were promulgated for the
\populus---a thing common to all \curationes.  For Cincinatus made
\dictator\ for the sake of waging war abdicated his \dictatura\
immediately once he had broken the enemy's forces and finished the war
in less than fifteen days.  \page{80b} %
Servilius Priscus \additio{did the same} by the eighth day; Mamercus
on the very same day he was named \dictator.  And he was named
\dictator\ by neither Senate nor \populus, nor likewise by the
\magistratuspl, or by proposal for the \populus, or by any \leges,
which were always necessary for the creation of \magistratuspl, but by
an interim \rex, who had but recently arisen from patrician blood: for
it was not enough to be a noble senator to name him \dictator.%
\mnote{Novus qui primus honorem in republica adeptus erat: nobilis
  noui hominis filius: patricius qui a patribus et conscriptis a
  Romulo stirpem ducebat.\footnotemark}[TODO] %
But if someone objects that Sulla was named a \dictator\ for eighty
years by the \emph{lex Valeria}, I shall repeat that argument of
Cicero's:%
\mnote{In lib.\ 1 de legibus.\footnotemark}[TODO] %
it was neither \dictatura\ nor a \lex, but the most cruel \tyrannis,
which he nevertheless renounced in the fourth year after he was made
\dictator---sc., when he had extinguished the flames of the civil wars
with the blood of the citizens (having preserved the intercession of
the Tribunes of the \plebs). \page{80c} And although Caesar had
invaded the perpetual \dictatura\ forty years later with the \libertas\
of the \plebs, he nonetheless left the Tribunes of the \plebs\ the
power to veto \platin{intercessio}.%
\footnote{An \lemph{intercessio} was a veto by a magistrate against
  some official act; the plebeian tribunes had such a veto in order to
  protect the \qq{interests of the \lemph{plebs} against abuses by
    magistrates}[,] though it could be (and was) used for less lofty
  purposes. Normally, this veto was unavailable against a
  \dictator. See \cite[506]{berger1953}.} %
However, \CHECK{before this}, % antea
during the solitary \consulatum\ of Pompey, when the name of
\dictatura\ had been removed from the \respublica, and, against the
\lex\ of Pompey, Caesar had undertaken to be made \dictator\ by the
\lemph{lex Servia}%
\footnote{TODO: \emph{actio Serviana}? Servian constitution?} %
he was slain in the middle of the Senate by a conspriacy of senators
\platin{principes}.  But let us grant that the highest \potestas, and
one freed from the \leges, outside of
\platin{extra} % or "outside of" or beyond or besides
a power to veto or appeal is given by the \populus\ to one person or
many: should we say that he \additio{or they} have a \iusmaiestatis?
For he has \maiestas\ who, after God immortal, sees no one greater
than himself.  Still, I firmly believe \platin{statuo} that there is
no \maiestas\ in them, but that they are bound to yield
\platin{deponere} the \imperium\ to the \populus\ from whom they have
the precarious \potestas\ when the fixed time has run its
course. % vel ad certum tempus, cuius temporis decursu
\page{80d} %
Nor is the \populus\ thought to have deprived itself of its \potestas,
even if it allots the highest \imperium---and one freed from the
\leges---to one or several persons; and even one not precarious
for a fixed time. In either way, he who had that highest \imperium\
is bound to render an account for the things he has done
to the \princeps\ or the \populus.
% NOTE: Knolles adds something here
But the \princeps\ or the \populus\ in whom lives \platin{inest}
\maiestas\ is not compelled to render to anyone an account of his
\additio{or its} deeds beyond God immortal.
%
What, therefore, if that highest \imperium\ be \concederi\ to one or
many people for ten years, as once there was one \emph{archon} among
the Athenians, whom they even called \Iudex? He was \praeesse\ of the
\respublica\ with the highest \potestas. Yet, \maiestas\ of the
\respublica\ was not within his control \platin{penes eum}, since
rather it was within the control of the \curator\ or \procurator\ of
the \populus, and he was bound to render an account of the deeds of
his rule \platin{imperii gesti}.  What if that highest (as I've said)
\potestas\ were bestowed on one or more people by that \lex\ for one
year so that he \additio{or they} be not compelled to render an
account \platin{rationem} of his \additio{or their} activities?  For
so the Cnidians did \page{81a} with sixty citizens each year, whom
they called Amymones, that is, \qq{mayors} \platin{maiores} without
any exception and censure. % omni exceptione ac reprehensione
Even so, there was no \maiestas\ of \imperium\ in them, since they
were bound to return the apportioned \platin{depositum} \imperium\ to
the \populus\ at the turn of the year. % anno vertente
Thus, we can call him highest \magistratus\ indeed, and a maximal
\potestas; \additio{but} we cannot call him a highest \princeps: for
one is a \dominus, another is a subject, another is a possessor%
\mnote{l.\ more, l.\ licet, l.\ \& quia de iuris.\ omnium iudicum, l.\
  1 de ofne eius cui mandat.\ est iurisd.\footnotemark}[TODO] %
and \proprietarius\ of \imperium, \additio{and} another can be called
 neither possessor of \imperium\ nor \dominus, but \custos.
We should come to the same conclusion about those whom the French
call \regentes\ of the \regnum, who tend to be created
due to the infancy, insanity, or absence of the \rex,
whether their \leges, edicts, rescripts, and, finally, all their decrees
are sanctioned by their name, hand, and seal
\page{81b} %
(as indeed was done by ancient custom \platin*{more maiorum},
before the \lex of Charles V, \Rex\ of the  French),
or the \regius\ seal and name of the \reges\
are proposed by \leges\ and \mandata.
For there is no (or virtually no) difference between the two since
what are done by a \procurator\ while the \dominus\ approves
seems to be done by the \dominus\ himself.%
\mnote{l.\ certe, §.\ 1, de precario. cap. muliers, de senten.\
  excommunicatorum.\footnotemark}[TODO] %


% page 81B (93) -- At rectores illi regii

\setsecnumdepth{part}
\englatglossary

\end{document}

%%% Local Variables:
%%% mode: latex
%%% TeX-master: t
%%% TeX-engine: xetex
%%% End:
