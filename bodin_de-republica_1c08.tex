\def\filedate{2014/10/10}
\def\fileversion{0.1}%
%
%
\newif\ifproof % <-- generic conditional for proof-reading hooks
\prooftrue
%
%
\documentclass[12pt,twoside]{memoir}

% low-level basics ===================================================
\usepackage{ifxetex,etoolbox}
\usepackage[svgnames]{xcolor}% load before bidi/polyglossia
\usepackage{xparse}


% My packages ========================================================
\usepackage{fonts+lang}
\usepackage{commands-generic}
\usepackage{page-setup}
\usepackage{terminology}
\usepackage{texts-law}


\begin{document}% ====================================================

\setcounter{chapter}{7}
\chapter{On the Right of Majesty}


\Maiestas\ is the \sumpot\ over citizens and subjects, and it is freed
from the laws,%
\footnote{Cf. \dig{1.3.31}.} %
which the Greeks call
  % switch to Greek input: C-x RET C-\ greek‐ibycus4
\grk{ἄκραν ἐξουσίαν},%
\footnote{Unlimited power?} %
\additio{or} sometimes %
\grk{κυρίαν ἀρχλὼ}% CHECK
\footnote{TODO} %
and \grk{κὺριον ωολέτευμα},% CHECK
\footnote{TODO} %
\mnote*{Foundation of the Commonwealth} %
the Italians \textitalian{\emph{Segnoria}},%
\footnote{\emph{Signoira} may be translated as lordship or
  dominion.} %
%
% HEBREW --- CHECK
\additio{and}
the Hebrews \texthebrew{.תרמר שכט}, % <-- missing one character !!
%
%
that is, \maiusimperium.  For \maiestas, says Festus, is named from
\qq{magnitude}[.]  \Maiestas\ should have been defined from the start,
which none of the philosophers or jurists did define, since it seems
that there is yet \page{79a} nothing greater or more necessary for
understanding the nature of the \Respublica.  And since we defined
\Respublica\ above
% where?
as a proper \platin{rectam} \gubernatio\ of several families and
things common among them with the highest and perpetual \potestas, it
should be explained why it comes from the appellation of \qq{highest
  and perpetual \potestas}[.]  We said it must be perpetual because
that a  \sumpot\ over \platin{in} citizens can come to be for
one \additio{person} or many; and yet it may not be allotted as
perpetual, but \additio{only} for a brief time: at the end of which,
they relinquish \platin{abdicant} the  \sumpot.  Therefore,
they cannot be called the highest \principes, but rather \custodes\ of
the  \sumpot\ and \imperium\ for so as long as the highest
\princeps\ or \populus\ reclaims the entrusted \platin{depositum}
\imperium\ of which they themselves are the truest possessors and
lords---\page{79b}%
not differently than those who gave their things for a loan or
as a pledge:%
%
\mnote{l.\ qui pignori, de usucap.; l.\ quod meo, de acquir.\ posess.%
 % \addtocounter{footnote}{-1}%
  \footnotemark}%
[\dig{41.3.33.4}(!); \hldig{41.2.18}] %
or, \additio{for as long as} they permitted their \iurisdictio\ or
\imperium\ to be enjoyed by another (whether for a certain time or for
a \lemph{precarium}%
\footnote{\dig{43.26.1 pr.}: \lemph{Precarium} is what is granted to
  one for him who seeks it with prayers to use for as long as he who
  granted it allows it.}%
% pr. Precarium est, quod precibus petenti utendum conceditur tamdiu,
% quamdiu is qui concessit patitur.
), they do not cease to be the arbiters and possessors of their
\potestas\ and \iurisdictio.%
\mnote{l.\ more.\ de iurisdic., l.\ et quia eod.\footnotemark}%
[\dig{2.1.5}; \hldig{2.1.6}.] %
Thus the Jurisconsult%
\footnote{This epithet usually refers to Ulpian in the same way
  Aristotle was known as \qq{the Philosopher}[.]} %
said that a prefect of Augustus is to return the entrusted \imperium\
once his magistracy has come to an end.% exacto magistratu
\mnote{l.\ 1, de offi.\ praefecti Augustalis\footnotemark}%
[\dig{1.17.1}.] %
Nor does it matter if a greater or lesser \potestas\ be assigned:
otherwise, if the highest \sumpot, granted as a \FIX{precarious
  bequest} of the \princeps\ were called \maiestas, he would be able
to use that power by his own \ius\ \platin{suo iure} against a
\princeps\ to whom nothing would remain beyond the empty name
\princeps: a servant, moreover, would command his lord.  Nothing more
absurd than that can be thought, for the person of the \princeps\ is
always excepted in every concession of \imperium\ to magistrates or
private individuals:%
\mnote{l.\ ult., qui satisdate; \CHECK{cor.\ ser., de pot. regia.}
  9. 17.\footnotemark}%
[\dig{2.8.16}.] %
however much \page{79c} \imperium\ there be that is assigned to
another, it is still less than what he has reserved to himself by his
\ius\ \maiestas.  Nor is it ever supposed that he has been despoiled
of his \sumpot; rather, it is supposed that he can take cognizance%
\mnote{l.\ Iudicium soluitur, de Iudic.\ l. solet de Iurisdic.} %
of those things which he assigned to the magistrates or
superintendents, either by anticipation, \CHECK{joint exercise}, or
\CHECK{summons}, and to take \additio{back} by force%
\mnote{Alexand.\ in l.\ ult. de iurisdict.; Pano. in cap.\ Pastoralis,
  de offi. ord.; Innocentius \& Felin.\ in cap. \CHECK{cum}
  ecclesiarum, eod.\footnotemark}[TODO] %
all \potestas\ \concessum\ to the magistrates.


From these \additio{reasons}, it happens that none have the
\iuramaiestatis: not the \Dictator{} of the Romans, not the Harmost of
Spartans, not the Esymnete of the Salonikans, not the Archus of
Maltese, not the Balia of the Florentines (when they enjoyed
\platin*{uterentur} popular \potestas), nor those who are called
regents of a \regnum\ by men today, % nostris hominibus
nor any other \magistratuspl\ or \curatores\ who have \sumpot\ (though
it be not allotted perpetually by concession of the \princeps\ or
\populus). \page{79d} Nor, even, did those ancient dicators have the
\sumius---even if they were appointed \platin{dicebantur} by the
\emph{noblest \lex}%
\footnote{TODO}%
---from whom one could not appeal,% provocari
\mnote{Festus in verbo optima lege. Plutar.\ in
  quaestion. Rom.\footnotemark}%
[TODO] % see also berger pdf331
and on whose creation all \magistratuspl\ resign \additio{their
  offices} until, \CHECK{by the sacred measure proposed [one] June},
the Tribunes of the \plebs\ were made sacred \custodes\ of popular
liberty, who, on the creation of the \Dictator{}, had the free \ius\
of intercession so that, if it were appealled from the \Dictator{},
the Tribunes could convene % cogerent
the \plebs, among whom there was contention % certabatur
about the appeal. % provocatio
For the \Dictator{} Papirius condemned to death Fabium Maximus I, and
Fabius Maximus II also condemned to death Minutius, \additio{both}
\magistri\ of the light cavalry \platin{minutium equitum}, because
they had fought against the command of the \Dictatores.  Even so, they
were freed on appeal \platin{ex provocatione} by judgment of the
\populus.  For, as Livy says:%
\mnote{9 lib.\ 7\footnotemark}[TODO] %
\qq{First, Fabius' father said, I call upon the Tribues, and I appeal
  to the \populus, who are more able than your \dictatura, to whom the
  \rex\ Tullus Hostilius yielded.}  It should be clear from these
words that he was neither \princeps\ nor chief \magistratus\ (as many
have thought), but a \curator\, whom they call our \page{80a}
\commissarius; nor that anything else was assigned to him beyond
taking charge of war-making \platin{curationem belli gerendi}, curbing
sedition, \CHECK{fixing the Republic} \platin{Reipublicae
  constituendae}, creating \magistratuspl, or \CHECK{helming the
  tiller} \platin{clavi figendi}.  \Maiestas, on the other hand, is
not delimited by a greater \potestas, any other \leges, or time.  But
not even the \emph{decemvirs} had a \iusmaiestatis\ for the
promulgation of \leges\ \platin{legum ferendarum}, even though they
had the \sumpot, freed also from the \leges, and even though all
\magistratuspl\ resign \additio{their office} at the creation of the
\emph{decemvirs}, since their \imperium\ takes its limit % finem
from the \leges\ of the Twelve Tables, which were promulgated for the
\populus---a thing common to all \curationes.  For Cincinatus made
\dictator\ for the sake of waging war abdicated his \dictatura\
immediately once he had broken the enemy's forces and finished the war
in less than fifteen days.  \page{80b} %
Servilius Priscus \additio{did the same} by the eighth day; Mamercus
on the very same day he was named \dictator.  And he was named
\dictator\ by neither Senate nor \populus, nor likewise by the
\magistratuspl, or by proposal for the \populus, or by any \leges,
which were always necessary for the creation of \magistratuspl, but by
an interim \rex, who had but recently arisen from patrician blood: for
it was not enough to be a noble senator to name him \dictator.%
\mnote{Novus qui primus honorem in republica adeptus erat: nobilis
  noui hominis filius: patricius qui a patribus et conscriptis a
  Romulo stirpem ducebat.\footnotemark}[TODO] %
But if someone objects that Sulla was named a \dictator\ for eighty
years by the \emph{lex Valeria}, I shall repeat that argument of
Cicero's:%
\mnote{In lib.\ 1 de legibus.\footnotemark}[TODO] %
it was neither \dictatura\ nor a \lex, but the most cruel \tyrannis,
which he nevertheless renounced in the fourth year after he was made
\dictator---sc., when he had extinguished the flames of the civil wars
with the blood of the citizens (having preserved the intercession of
the Tribunes of the \plebs). \page{80c} And although Caesar had
invaded the perpetual \dictatura\ forty years later with the \libertas\
of the \plebs, he nonetheless left the Tribunes of the \plebs\ the
power to veto \platin{intercessio}.%
\footnote{An \lemph{intercessio} was a veto by a magistrate against
  some official act; the plebeian tribunes had such a veto in order to
  protect the \qq{interests of the \lemph{plebs} against abuses by
    magistrates}[,] though it could be (and was) used for less lofty
  purposes. Normally, this veto was unavailable against a
  \dictator. See \cite[506]{berger1953}.} %
However, \CHECK{before this}, % antea
during the solitary \consulatum\ of Pompey, when the name of
\dictatura\ had been removed from the \respublica, and, against the
\lex\ of Pompey, Caesar had undertaken to be made \dictator\ by the
\lemph{lex Servia}%
\footnote{TODO: \emph{actio Serviana}? Servian constitution?} %
he was slain in the middle of the Senate by a conspriacy of senators
\platin{principes}.  But let us grant that the highest \potestas, and
one freed from the \leges, outside of
\platin{extra} % or "outside of" or beyond or besides
a power to veto or appeal is given by the \populus\ to one person or
many: should we say that he \additio{or they} have a \iusmaiestatis?
For he has \maiestas\ who, after God immortal, sees no one greater
than himself.  Still, I firmly believe \platin{statuo} that there is
no \maiestas\ in them, but that they are bound to yield
\platin{deponere} the \imperium\ to the \populus\ from whom they have
the precarious \potestas\ when the fixed time has run its
course. % vel ad certum tempus, cuius temporis decursu
\page{80d} %
Nor is the \populus\ thought to have deprived itself of its \potestas,
even if it allots the highest \imperium---and one freed from the
\leges---to one or several persons; and even one not precarious
for a fixed time. In either way, he who had that highest \imperium\
is bound to render an account for the things he has done
to the \princeps\ or the \populus.
% NOTE: Knolles adds something here
But the \princeps\ or the \populus\ in whom lives \platin{inest}
\maiestas\ is not compelled to render to anyone an account of his
\additio{or its} deeds beyond God immortal.
%
What, therefore, if that highest \imperium\ be \concederi\ to one or
many people for ten years, as once there was one \emph{archon} among
the Athenians, whom they even called \Iudex? He was \praeesse\ of the
\respublica\ with the highest \potestas. Yet, \maiestas\ of the
\respublica\ was not within his control \platin{penes eum}, since
rather it was within the control of the \curator\ or \procurator\ of
the \populus, and he was bound to render an account of the deeds of
his rule \platin{imperii gesti}.




% bottom of page 80D (92); knolles pdf97i
\end{document}

But supposing the king grants absolute power to a lieutenant for the term
of his life, is not that a perpetual sovereign power? For if one confines
perpetual to that which has no termination whatever, then sovereignty cannot
subsist save in aristocracies and popular states, which never die. If one is
to include monarchy too, sovereignty must be vested not in the king alone,
but in the king and the heirs of his body, which supposes a strictly
hereditary monarchy. In that case there can be very few sovereign kings,
since there are only a very few strictly hereditary monarchies. Those
especially who come to the throne by election could not be included.

A perpetual authority therefore must be understood to mean one that lasts
for the lifetime of him who exercises it. If a sovereign magistrate is
given office for one year, or for any other predetermined period, and
continues to exercise the authority bestowed on him after the conclusion of
his term, he does so either by consent or by force and violence. If he does
so by force, it is manifest tyranny. The tyrant is a true sovereign for all
that. The robber's possession by violence is true and natural possession
although contrary to the law, for those who were formerly in possession have
been disseized. But if the magistrate continues in office by consent, he is
not a sovereign prince, seeing that he only exercises power on sufferance.
Still less is he a sovereign if the term of his office is not fixed, for in
that case he has no more than a precarious commission. ...

What bearing have these considerations on the case of the man to whom the
people has given absolute power for the term of his natural life? One must
distinguish. If such absolute power is given him simply and
unconditionally, and not in virtue of some office or commission, nor in the
form of a revocable grant, the recipient certainly is, and should be
acknowledged to be, a sovereign. The people has renounced and alienated its
sovereign power in order to invest him with it and put him in possession,
and it thereby transfers to him all its powers, authority, and sovereign
rights, just as does the man who gives to another possessory and proprietary
rights over what he formerly owned. The civil law expresses this in the
phrase 'all power is conveyed to him and vested in him'.[4]

But if the people give such power for the term of his natural life to
anyone as its official or lieutenant, or only gives the exercise of such
power, in such a case he is not a sovereign, but simply an officer,
lieutenant, regent, governor, or agent, and as such has the exercise only of
a power inhering in another. When a magistrate institutes a perpetual
lieutenant, even if he abandons all his rights of jurisdiction and leaves
their exercise entirely to his lieutenant, the authority to command and to
judge nevertheless does not reside in the lieutenant, nor the action and
force of the law derive from him. If he exceeds his authority his acts have
no validity, unless approved and confirmed by him from whom he draws his
authority. For this reason King John, after his return from captivity in
England, solemnly ratified all the acts of his son Charles, who had acted in
his name as regent, in order, as was necessary, to regularize the position.

Whether then one exercises the power of another by commission, by
institution, or by delegation, or whether such exercise is for a set term,
or in perpetuity, such a power is not a sovereign power, even if there is no
mention of such words as representative, lieutenant, governor, or regent,
in the letters of appointment, or even if such powers are a consequence of
the normal working of the laws of the country. In ancient times in Scotland,
for instance, the law vested the entire governance of the realm in the next
of kin, if the king should be a minor, on condition that everything that was
done, was done in the king's name. But this law was later altered because of
its inconvenient consequences.

Let us now turn to the other term of our definition and consider the force
of the word absolute. The people or the magnates of a commonwealth can bestow
simply and unconditionally upon someone of their choice a sovereign and
perpetual power to dispose of their property and persons, to govern the
state as he thinks fit, and to order the succession, in the same way that
any proprietor, out of his liberality, can freely and unconditionally make a
gift of his property to another. Such a form of gift, not being qualified in
any way, is the only true gift, being at once unconditional and irrevocable.
Gifts burdened with obligations and hedged with conditions are not true
gifts. Similarly sovereign power given to a prince charged with conditions
is neither properly sovereign, nor absolute, unless the conditions of
appointment are only such as are inherent in the laws of God and of nature.
...

If we insist however that absolute power means exemption from all law
whatsoever, there is no prince in the world who can be regarded as
sovereign, since all the princess of the earth are subject to the laws of
God and of nature, and even to certain human laws common to all nations. On
the other hand, it is possible for a subject who is neither a prince nor a
ruler, to be exempted from all the laws, ordinances, and customs of the
commonwealth. We have an example in Pompey the Great who was dispensed from
the laws for five years, by express enactment of the Roman people, at the
instance of the Tribune Gabinius ... But notwithstanding such exemptions
from the operations of the law, the subject remains under the authority of
him who exercises sovereign power, and owes him obedience.

On the other hand it is the distinguishing mark of the sovereign that he
cannot in any way be subject to the commands of another, for it is he who
makes law for the subject, abrogates law already made, and amends obsolete
law. No one who is subject either to the law or to some other person can do
this. That is why it is laid down in the civil law that the prince is above
the law, for the word law in Latin implies the command of him who is
invested with sovereign power. Therefore we find in all statutes the phrase
'notwithstanding all edicts and ordinances to the contrary that we have
infringed, or do infringe by these present'. This clause applies both to
former acts of the prince himself, and to those of his predecessors. For all
laws, ordinances, letters patent, privileges, and grants whatsoever issued
by the prince, have force only during his own lifetime, and must be
expressly, or at least tacitly, confirmed by the reigning prince who has
cognizance of them ... In proof of which, it is the custom of this realm for
all corporations and corporate bodies to ask for the confirmation of their
privileges, rights, and jurisdictions, on the accession of a new king. Even
Parlements and high courts do this, as well as individual officers of the
crown.

If the prince is not bound by the laws of his predecessors, still less can
he be bound by his own laws. One may be subject to laws made by another, but
it is impossible to bind oneself in any matter which is the subject of one's
own free exercise of will. As the law says, 'there can be no obligation in
any matter which proceeds from the free will of the undertaker'.[5] It
follows of necessity that the king cannot be subject to his own laws. Just
as, according to the canonists, the Pope can never tie his own hands, so the
sovereign prince cannot bind himself, even if he wishes. For this reason
edicts and ordinances conclude with the formula 'for such is our good
pleasure', thus intimating that the laws of a sovereign prince, even when
founded on truth and right reason, proceed simply from his own free will.

It is far otherwise with divine and natural laws. All the princes of the
earth are subject to them, and cannot contravene them without treason and
rebellion against God. His yoke is upon them, and they must bow their heads
in fear and reverence before His divine majesty. The absolute power of
princes and sovereign lords does not extend to the laws of God and of
nature. He who best understood the meaning of absolute power, and made kings
and emperors submit to his will, defined his sovereignty as a power to
override positive law; he did not claim power to set aside divine and
natural law.[6]

But supposing the prince should swear to keep the laws and customs of his
country, is he not bound by that oath? One must distinguish. If a prince
promises in his own heart to obey his own laws, he is nevertheless not
bound to do so, any more than anyone is bound by an oath taken to himself.
Even private citizens are not bound by private oaths to keep agreements. The
law permits them to cancel them, even if the agreements are in themselves
reasonable and good. But if one sovereign prince promises another sovereign
prince to keep the agreements entered into by his predecessors, he is bound
to do so even if not under oath, if that other prince's interests are
involved. If they are not, he is not bound either by a promise, or even by
an oath.

The same holds good of promises made by the sovereign to the subject, even
if the promises were made prior to his election (for this does not make the
difference that many suppose). It is not that the prince is bound either by
his own laws or those of his predecessors. But he is bound by the just
covenants and promises he has made, whether under oath to do so or not, to
exactly the same extent that a private individual is bound in like case. A
private individual can be released from a promise that was unjust or
unreasonable, or beyond his competence to fulfil, or extracted from him by
misrepresentations or fraud, or made in error, or under restraint and by
intimidation, because of the injury the keeping of it does him. In the same
way a sovereign prince can make good any invasion of his sovereign rights,
and for the same reasons. So the principle stands, that the prince is not
subject to his own laws, or those of his predecessors, but is bound by the
just and reasonable engagements which touch the interests of his subjects
individually or collectively.

Many have been led astray by confusing the laws of the prince with covenants
entered into by him. This confusion has led some to call these covenants
contractual laws. This is the term used in Aragon when the king issues an
ordinance upon the petition of the Estates, and in return receives some aid
or subsidy. It is claimed that he is strictly bound by these laws, even
though he is not by any of his other enactments. It is however admitted that
he may override even these when the purpose of their enactment no longer
holds. All this is true enough, and well-founded in reason and authority.
But no bribe or oath is required to bind a sovereign prince to keep a law
which is in the interests of his subjects. The bare word of a prince should
be as sacred as a divine pronouncement. It loses its force if he is
ill-thought of as one who cannot be trusted except under oath, nor relied
on to keep a promise unless paid to do so. Nevertheless it remains true in
principle that the sovereign prince can set aside the laws which he has
promised or sworn to observe, if they no longer satisfy the requirements of
justice, and he may do this without the consent of his subjects. It should
however be added that the abrogation must be express and explicit in its
reference, and not just in the form of a general repudiation. But if on the
other hand there is no just cause for breaking a law which the prince has
promised to keep, the prince ought not to do so, and indeed cannot
contravene it, though he is not bound to the same extent by the promises
and covenants of his predecessors unless he succeeds by strict hereditary
right.

A law and a covenant must therefore not be confused. A law proceeds from him
who has sovereign power, and by it he binds the subject to obedience, but
cannot bind himself. A covenant is a mutual undertaking between a prince and
his subjects, equally binding on both parties, and neither can contravene it
to the prejudice of the other, without his consent. The prince has no
greater privilege than the subject in this matter. But in the case of laws,
a prince is no longer bound by his promise to keep them when they cease to
satisfy the claims of justice. Subjects however must keep their engagements
to one another in all circumstances, unless the prince releases them from
such obligations. Sovereign princes are not bound by oath to keep the laws
of their predecessors. If they are so bound, they are not properly speaking
sovereign. ...

The constitutional laws of the realm, especially those that concern the
king's estate being, like the salic law, annexed and united to the Crown,
cannot be infringed by the prince. Should he do so, his successor can always
annul any act prejudicial to the traditional form of the monarchy,[7] since
on this is founded and sustained his very claim to sovereign majesty. ...

As for laws relating to the subject, whether general or particular, which
do not involve any question of the constitution, it has always been usual
only to change them with the concurrence of the three estates, either
assembled in the States-General of the whole of France, or in each bailiwick
separately. Not that the king is bound to take their advice, or debarred
from acting in a way quite contrary to what they wish, if his acts are based
on justice and natural reason. At the same time the majesty of the prince is
most fully manifested in the assembly of the three estates of the whole
realm, humbly petitioning and supplicating him, without any power of
commanding or determining, or any right to a deliberative voice. Only that
which it pleases the prince to assent to or dissent from, to command or to
forbid, has the force of law and is embodied in his edict or ordinance.

Those who have written books about the duties of magistrates and such like
matters[8] are in error in maintaining that the authority of the Estates is
superior to that of the prince. Such doctrines serve only to encourage
subjects to resist their sovereign rulers. Besides, such views bear no
relation to the facts, except when the king is in captivity, lunatic or a
minor. If he were normally subject to the Estates, he would be neither a
prince nor a sovereign, and the commonwealth would not be a kingdom or a
monarchy, but a pure aristocracy where authority is shared equally between
the members of the ruling class. ...

Although in the Parliaments of the kingdom of England, which meet every
three years, all three orders use great freedom of speech, as is
characteristic of northern peoples, they still must proceed by petitions and
supplications ... Moreover Parliaments in England can only assemble, as in
this kingdom and in Spain, under letters patent expressly summoning them in
the king's name. This is sufficient proof that Parliaments have no
independent power of considering, commanding or determining, seeing that
they can neither assemble nor adjourn without express royal command ... It
may be objected that no extraordinary taxes or subsidies can be imposed
without the agreement and consent of Parliament. King Edward I agreed to
this principle in the Great Charter, which is always appealed to by the
people against the claims of the king. But I hold that in this matter no
other king has any more right than has the King of England, since it is not
within the competence of any prince in the world to levy taxes at will on
his people, or seize the goods of another arbitrarily, as Philippe de
Comines very wisely argued at the Estates at Tours, as we may read in his
Memoirs.[9]

We must agree then that the sovereignty of the king is in no wise qualified
or diminished by the existence of Estates. On the contrary his majesty
appears more illustrious when formally recognized by his assembled subjects,
even though in such assemblies princes, not wishing to fall out with their
people, agree to many things which they would not have consented to, unless
urged by the petitions, prayers, and just complaints of a people burdened by
grievances unknown to the prince. After all, he depends for his information
on the eyes and ears and reports of others.

From all this it is clear that the principal mark of sovereign majesty and
absolute power is the right to impose laws generally on alt subjects
regardless of their consent ... And if it is expedient that if he is to
govern his state well, a sovereign prince must be above the law, it is even
more expedient that the ruling class in an aristocracy should be so, and
inevitable in a popular state. A monarch in a kingdom is set apart from his
subjects, and the ruling class from the people in an aristocracy. There are
therefore in each case two parties, those that rule on the one hand, and
those that are ruled on the other. This is the cause of the disputes about
sovereignty that arise in them, but cannot in a popular state ... There the
people, rulers and ruled, form a single body and so cannot bind themselves
by their own laws. ...

When edicts are ratified by Estates or Parlements, it is for the purpose of
securing obedience to them, and not because otherwise a sovereign prince
could not validly make law. As Theodosius said with reference to the consent
of the Senate, 'it is not a matter of necessity but of expediency'. He also
remarked that it was most becoming in a sovereign prince to keep his own
laws, for this is what makes him feared and respected by his subjects,
whereas nothing so undermines his authority as contempt for them. As a Roman
Senator observed 'it is more foolish and ill-judged to break your own laws
than those of another'.

But may it not be objected that if the prince forbids a sin, such as
homicide, on pain of death, he is in this case bound to keep his own law The
answer is that this is not properly the prince's own law, but a law of God
and nature, to which he is more strictly bound than any of his subjects.
Neither his council, nor the whole body of the people, can exempt him from
his perpetual responsibility before the judgement-seat of God, as Solomon
said in unequivocal terms. Marcus Aurelius also observed that the magistrate
is the judge of persons, the prince of the magistrates, and God of the
prince. Such was the opinion of the two wisest rulers the world has ever
known. Those who say without qualification that the prince is bound neither
by any law whatsoever, nor by his own express engagements, insult the
majesty of God, unless they intend to except the laws of God and of nature,
and all just covenants and solemn agreements. Even Dionysius, tyrant of
Syracuse, said to his mother that he could exempt her from the laws and
customs of Syracuse, but not from the laws of God and of nature. For just as
contracts and deeds of gift of private individuals must not derogate from
the ordinances of the magistrate, nor his ordinances from the law of the
land, nor the law of the land from the enactments of a sovereign prince, so
the laws of a sovereign prince cannot override or modify the laws of God and
of nature. ...

There is one other point. If the prince is bound by the laws of nature, and
the civil law is reasonable and equitable, it would seem to follow that the
prince is also bound by the civil law. As Pacatius said to the Emperor
Theodosius 'as much is permitted to you as is permitted by the laws'. In
answer to this I would point out that the laws of a sovereign prince concern
either public or private interests or both together. All laws moreover can
be either profitable at the expense of honour, or profitable without
involving honour at all, or honourable without profit, or neither honourable
nor profitable. When I say 'honour' I mean that which conforms with what is
natural and right, and it has already been shown that the prince is bound in
such cases. Laws of this kind, though published by the prince's authority,
are properly natural laws. Laws which are profitable as well as just are
even more binding on him. One need hardly concern oneself about the sanctity
of laws which involve neither profit nor honour. But if it is a question of
weighing honour against profit, honour should always be preferred. Aristides
the Just said of Themistocles that his advice was always very useful to the
people, but shameful and dishonourable.

But if a law is simply useful and does not involve any principle of natural
justice, the prince is not bound by it, but can amend it or annul it
altogether as he chooses, provided that with the alteration of the law the
profit to some does not do damage to others without just cause. The prince
then can annul an ordinance which is merely useful in order to substitute
one more or less advantageous, for profit, honour, and justice all have
degrees of more and less. And just as the prince can choose the most useful
among profitable laws, so he can choose the most just among equitable laws,
even though while some profit by them others suffer, provided it is the
public that profits, and only the private individual that suffers. It is
however never proper for the subject to disobey the laws of the prince under
the pretext that honour and justice require it. ...

Edicts and ordinances therefore do not bind the ruler except in so far as
they embody the principles of natural justice; that ceasing, the obligation
ceases. But subjects are bound till the ruler has expressly abrogated the
law, for it is a law both divine and natural that we should obey the edicts
and ordinances of him whom God has set in authority over us, providing his
edicts are not contrary to God's law. For just as the rear-vassal owes an
oath of fealty in respect of and against all others, saving his sovereign
prince, so the subject owes allegiance to his sovereign prince in respect of
and against all others, saving the majesty of God, who is lord of all the
princes of this world. From this principle we can deduce that other rule,
that the sovereign prince is bound by the covenants he makes either with his
subjects, or some other prince. Just because he enforces the covenants and
mutual engagements entered into by his subjects among themselves, he must be
the mirror of justice in all his own acts ... He has a double obligation in
this case. He is bound in the first place by the principles of natural
equity, which require that conventions and solemn promises should be kept,
and in the second place in the interests of his own good faith, which he
ought to pre-serve even to his own disadvantage, because he is the formal
guarantor to all his subjects of the mutual faith they owe one another. ...

A distinction must therefore be made between right and law, for one implies
what is equitable and the other what is commanded. Law is nothing else than
the command of the sovereign in the exercise of his sovereign power. A
sovereign prince is not subject to the laws of the Greeks, or any other
alien power, or even those of the Romans, much less to his own laws, except
in so far as they embody the law of nature which, according to Pindar, is
the law to which all kings and princes are subject. Neither Pope nor Emperor
is exempt from this law, though certain flatterers say they can take the
goods of their subjects at will. But both civilians and canonists have
repudiated this opinion as contrary to the law of God. They err who assert
that in virtue of their sovereign power princes can do this. It is rather
the law of the jungle, an act of force and violence. For as we have shown
above, absolute power only implies freedom in relation to positive laws, and
not in relation to the law of God. God has declared explicitly in His Law
that it is not just to take, or even to covet, the goods of another. Those
who defend such opinions are even more dangerous than those who act on them.
They show the lion his claws, and arm princes under a cover of just claims.
The evil will of a tyrant, drunk with such flatteries, urges him to an abuse
of absolute power and excites his violent passions to the pitch where
avarice issues in confiscations, desire in adultery, and anger in murder.
...

Since then the prince has no power to exceed the laws of nature which God
Himself, whose image he is, has decreed, he cannot take his subjects'
property without just and reasonable cause, that is to say by purchase,
exchange, legitimate confiscation, or to secure peace with the enemy when
it cannot be otherwise achieved. Natural reason instructs us that the public
good must be preferred to the particular, and that subjects should give up
not only their mutual antagonisms and animosities, but also their
possessions, for the safety of the commonwealth. ...

It remains to be determined whether the prince is bound by the covenants of
his predecessors, and whether, if so, it is a derogation or his sovereign
power ... A distinction must be made between the ruler who succeeds because
he is the natural heir of his predecessor, and the ruler who succeeds in
virtue of the laws and customs of the realm. In the first case the heir is
bound by the oaths and promises of his predecessors just as is any ordinary
heir. In the second case he is not so bound even if he is sworn, for the
oath of the predecessor does not bind the successor. He is bound however in
all that tends to the benefit of the kingdom.

There are those who will say that there is no need of such distinctions
since the prince is bound in any case by the law of nations, under which
covenants are guaranteed. But I consider that these distinctions are
necessary nevertheless, since the prince is bound as much by the law of
nations, but no more, than by any of his own enactments. If the law of
nations is iniquitous in any respect, he can disallow it within his own
kingdom, and forbid his subjects to observe it, as was done in France in
regard to slavery. He can do the same in relation to any other of its
provisions, so long as he does nothing against the law of God. If justice is
the end of the law, the law the work of the prince, and the prince the image
of God, it follows of necessity that the law of the prince should be
modelled on the law of God.




%%% Local Variables:
%%% mode: latex
%%% TeX-master: t
%%% TeX-engine: xetex
%%% End:
